\subsection{Carreras}
La Unidad Profesional Interdisciplinaria de Ingeniería y Ciencias Sociales y Administrativas (UPIICSA) ofrece programas académicos en las áreas de ingeniería, ciencias administrativas y sociales. Estas carreras se orientan a formar profesionales capaces de integrar el conocimiento técnico con una visión organizacional, fomentando la innovación y la solución de problemas reales en distintos sectores productivos.

\subsubsection{Carreras del área de Ingeniería}

\subsubsection{Ingeniería en Informática}
Forma profesionistas capaces de diseñar, desarrollar e implementar sistemas de información y soluciones tecnológicas para la gestión de datos y procesos empresariales.
\begin{itemize}
	\item \textbf{Duración:} 9 semestres  
	\item \textbf{Enfoque:} Desarrollo de software, bases de datos, redes y seguridad informática.  
	\item \textbf{Salidas profesionales:} Analista de sistemas, desarrollador de software, administrador de TI.
\end{itemize}

\subsubsection{Ingeniería Industrial}
Orienta al estudiante en la optimización de recursos humanos, materiales y tecnológicos dentro de las organizaciones, promoviendo la productividad y la eficiencia.
\begin{itemize}
	\item \textbf{Duración:} 9 semestres  
	\item \textbf{Enfoque:} Producción, calidad, logística y mejora continua.  
	\item \textbf{Salidas profesionales:} Ingeniero de procesos, consultor de calidad, gerente de producción.
\end{itemize}

\subsubsection{Ingeniería en Transporte}
Forma profesionales especializados en la planeación, operación y gestión de sistemas de transporte de personas y mercancías, considerando aspectos técnicos, económicos y ambientales.
\begin{itemize}
	\item \textbf{Duración:} 9 semestres  
	\item \textbf{Enfoque:} Logística, diseño de rutas, seguridad vial y movilidad sostenible.  
	\item \textbf{Salidas profesionales:} Planificador de transporte, gestor logístico, analista de movilidad.
\end{itemize}

\subsubsection{Ingeniería en Sistemas Automotrices}
Prepara ingenieros con conocimientos en el diseño, mantenimiento y diagnóstico de sistemas automotrices modernos, integrando mecánica, electrónica y control.
\begin{itemize}
	\item \textbf{Duración:} 9 semestres  
	\item \textbf{Enfoque:} Tecnología vehicular, sistemas eléctricos, motores y manufactura.  
	\item \textbf{Salidas profesionales:} Ingeniero automotriz, técnico en diagnóstico vehicular, desarrollador de sistemas mecatrónicos.
\end{itemize}

\subsection{Carreras del área de Ciencias Sociales y Administrativas}

\subsubsection{Licenciatura en Administración Industrial}
Forma profesionales con la capacidad de gestionar procesos administrativos, financieros y operativos dentro de las organizaciones, integrando herramientas tecnológicas y de gestión.
\begin{itemize}
	\item \textbf{Duración:} 8 semestres  
	\item \textbf{Enfoque:} Planeación estratégica, recursos humanos, finanzas y producción.  
	\item \textbf{Salidas profesionales:} Administrador de operaciones, jefe de área, analista de negocios.
\end{itemize}

\subsubsection{Licenciatura en Ciencias de la Informática}
Prepara profesionistas con una sólida formación en el desarrollo, análisis y gestión de sistemas informáticos aplicados a los procesos administrativos y de decisión organizacional.
\begin{itemize}
	\item \textbf{Duración:} 9 semestres  
	\item \textbf{Enfoque:} Programación, análisis de datos, inteligencia de negocios y gestión tecnológica.  
	\item \textbf{Salidas profesionales:} Analista de datos, desarrollador de software empresarial, consultor en TI.
\end{itemize}

\subsubsection{Licenciatura en Relaciones Comerciales}
Capacita profesionales especializados en mercadotecnia, comercio internacional y ventas, con una visión integral de los mercados nacionales y globales.
\begin{itemize}
	\item \textbf{Duración:} 8 semestres  
	\item \textbf{Enfoque:} Mercadotecnia, negocios internacionales, publicidad y gestión comercial.  
	\item \textbf{Salidas profesionales:} Ejecutivo de ventas, especialista en marketing, analista de comercio exterior.
\end{itemize}
