\section{Unity}
Unity es una plataforma de desarrollo 3D en tiempo real para compilar aplicaciones 2D y 3D, 
como juegos y simulaciones, con .NET y el lenguaje de programación C#.
El impacto que tiene en el desarrollo de aplicaciones AR es significativo, 
ya que proporciona un entorno robusto y versátil para crear experiencias inmersivas y atractivas.
\subsection{Ventajas de usar Unity para AR}
Unity ofrece varias ventajas para el desarrollo de aplicaciones de realidad aumentada:
\begin{itemize}
    \item \textbf{Compatibilidad multiplataforma:} 
    Unity permite desarrollar aplicaciones AR que pueden ejecutarse en múltiples plataformas, como iOS, Android, entre otros, facilitando la distribución y el acceso a una audiencia más amplia.
    \item \textbf{Herramientas integradas:} 
    Unity proporciona herramientas específicas para AR, como AR Foundation, que simplifican la integración de funcionalidades de AR en las aplicaciones.
    \item \textbf{Comunidad activa:} 
    La amplia comunidad de desarrolladores de Unity ofrece recursos, tutoriales y soporte, lo que facilita la resolución de problemas y el aprendizaje continuo.
    \item \textbf{Gráficos avanzados:} 
    Unity ofrece capacidades gráficas avanzadas que permiten crear experiencias visualmente atractivas y realistas en aplicaciones AR.
    \item \textbf{Facilidad de uso:} 
    La interfaz intuitiva de Unity y su sistema de arrastrar y soltar facilitan el desarrollo rápido de prototipos y la iteración en el diseño de aplicaciones AR.
\end{itemize}
\subsection{Limitaciones de Unity para AR}
A pesar de sus ventajas, Unity también presenta algunas limitaciones en el desarrollo de aplicaciones AR, en este caso en la navegación de interiores.
\begin{itemize}
    \item \textbf{Precisión en la localización:}
    Unity no tiene herramientas nativas de su entorno para la localización precisa en interiores, lo que puede ser un desafío para aplicaciones de navegación.
    \item \textbf{Curva de aprendizaje:} 
    Aunque Unity es accesible, los desarrolladores nuevos pueden enfrentar una curva de aprendizaje significativa, especialmente si no están familiarizados con C# o el desarrollo 3D.
    \item \textbf{Rendimiento:} 
    Las aplicaciones AR pueden ser exigentes en términos de rendimiento, y optimizar las aplicaciones en Unity para dispositivos móviles puede ser un desafío.
    \item \textbf{Consumo de bateria:} 
    Las aplicaciones AR desarrolladas en Unity pueden consumir una cantidad significativa de batería en dispositivos móviles, lo que puede afectar la experiencia del usuario.
    \item \textbf{Datos en tiempo real:}
    Unity no proporciona soluciones integradas para la gestión de datos en tiempo real, 
    lo que puede ser necesario para aplicaciones de navegación en interiores que requieren actualizaciones constantes.
\end{itemize}
