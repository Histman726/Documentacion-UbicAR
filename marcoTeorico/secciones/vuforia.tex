\section{Vuforia}
Vuforia AR es una plataforma de desarrollo de realidad aumentada que permite crear aplicaciones capaces de reconocer imágenes, objetos, superficies y entornos del mundo real para superponer sobre ellos elementos digitales como modelos 3D, videos, texto o animaciones. Utiliza visión por computadora para rastrear con precisión los objetos y mantener el contenido virtual alineado con el entorno físico, lo que facilita experiencias interactivas en áreas como educación, industria, publicidad y entretenimiento.

\begin{figure}[H]
	\centering
	\includegraphics[width=0.60\textwidth]{Vuforia concepto.jpg}
	\caption{Plataforma de Vuforia para la Realidad Aumentada}
	\label{fig:concepto-Vuforia}
\end{figure}

\subsection{Funcionalidades}
\begin{itemize}
    \item \textbf{Image Targets:} 
    Vuforia puede reconocer y rastrear imágenes planas, conocidas como marcadores, para superponer contenido digital en el mundo real.
    \item \textbf{Cloud Image Recognition:} 
    Reconocer un gran conjunto de imágenes y actualizar con frecuencia la base de datos con nuevas imágenes.
    \textbf{Bases de datos en la nube frente a dispositivos:} 
    Conozca las diferencias entre los dos tipos de bases de datos y seleccione la solución que mejor se adapte a sus necesidades.
    \textbf{Servicios web de Vuforia:} 
    Con la API de VWS, puede administrar estas grandes bases de datos de imágenes en la nube de manera eficiente y puede automatizar sus flujos de trabajo en su propio sistema de administración de contenido.
    \item \textbf{Model Target:} 
    Permite reconocer objetos por forma utilizando modelos 3D preexistentes. 
    Coloque contenido de realidad aumentada en una amplia variedad de artículos como equipos industriales, vehículos, juguetes y electrodomésticos.
    \item \textbf{Ground Plane:} 
    Permite colocar contenido en superficies horizontales del entorno, como mesas y suelos.
    \item \textbf{VuMarks:} 
    Estos son marcadores personalizados que pueden codificar una variedad de formatos de datos. Admiten tanto la identificación única como el seguimiento de aplicaciones de RA.
    \item \textbf{Integración con Unity:} 
    Vuforia se integra fácilmente con Unity, lo que facilita el desarrollo de aplicaciones AR utilizando las herramientas y capacidades de Unity.
    \item \textbf{Cámara externa:}
    Accede a los datos de video desde una cámara fuera de la de un teléfono o tableta al crear experiencias de realidad aumentada. 
    La cámara externa se utiliza en extensión del marco de controladores.
    \item \textbf{Vuforia Fusion:}
    Diseñado para proporcionar la mejor experiencia de RA posible en una amplia gama de dispositivos. 
    Fusion detecta las capacidades del dispositivo subyacente (como ARKit/ARCore) y las fusiona con las funciones de Vuforia Engine, 
    lo que permite a los desarrolladores confiar en una única API de Vuforia para una experiencia de realidad aumentada óptima.
    \item \textbf{Grabación  y reproducción:}
    Grabe y reproduzca su sesión de RA para probar, experimentar y mejorar su flujo de trabajo de desarrollo de RA con la grabadora de sesiones. 
    Use la API o el GameObject SessionRecorder listo para usar en Unity para registrar sus destinos de Vuforia y continuar desarrollando incluso cuando esos destinos o espacios no estén disponibles.
\end{itemize}

\subsection{Comparación con otros SDKs}
\begin{itemize}
    \item \textbf{ARKit y ARCore:} 
    Mientras que ARKit (iOS) y ARCore (Android) se centran en el reconocimiento de superficies y la detección de planos, Vuforia destaca en el reconocimiento de imágenes y objetos, ofreciendo una mayor variedad de opciones para desarrolladores.
    \item \textbf{Wikitude:} 
    Wikitude es otro SDK popular que ofrece funcionalidades similares a Vuforia. Sin embargo, Vuforia es conocido por su facilidad de uso y su integración con Unity, lo que lo hace más accesible para desarrolladores principiantes.
    \item \textbf{EasyAR:} 
    EasyAR es una alternativa más económica a Vuforia, pero puede carecer de algunas funcionalidades avanzadas que Vuforia ofrece, como el soporte para Model Targets y VuMarks.
    \item \textbf{Kudan:}
    Kudan es otro SDK de RA que se centra en el reconocimiento de imágenes y el seguimiento. Sin embargo, Vuforia ofrece una gama más amplia de funcionalidades y una comunidad de desarrolladores más grande.
\end{itemize}