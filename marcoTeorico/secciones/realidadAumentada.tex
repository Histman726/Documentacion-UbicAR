\section{Realidad aumentada}
La realidad aumentada se refiere a la integración en tiempo real de información díggital
en el entorno del usuario. La tecnología de realidad aumentada superpone elementos virtuales,
enrequieciendo la percepción del mundo real. Esta tecnología se utiliza en diversas aplicaciones,
desde juegos y entretenimiento hasta educación, medicina y comercio. 
La realidad aumentada puede implementarse a través de dispositivos como smartphones, tabletas, gafas inteligentes y cascos de realidad aumentada, 
permitiendo a los usuarios interactuar con el contenido digital de manera intuitiva y enriquecedora.
\subsection {Evolución de la realidad aumentada} 
\subsection {Componentes de la realidad aumentada}
\subsection{Tipos de realidad aumentada}
	Hay dos tipos fundamentales de realidad aumentada: \\
	\textbf {Por marcadores.} \\
	Superpone contenido dígital a un activador físico (es decir, un marcador), que puede ser una imagen, un objeto o un código QR.
	Cuando la cámara del dispositivos detecta el marcador, el software de realidad aumentada reconoce el patrón y proyecta el contenido virtual en la posición del marcador.
	Este tipo de realidad aumentada es común en aplicaciones educativas y de entretenimiento, 
	donde los usuarios pueden interactuar con modelos 3D, animaciones o información adicional al escanear un marcador específico.
	Dado que se pueda acceder a este tipo de realidad aumentada en cualquier momento y desde cualqiuer dispositivo,
	es el modelo de realidad aumentada más flexible y accesible. \\
	\textbf {Sin marcadores.} \\
	En contraparte, no requiere de un activador físico. Este tipo se basa en sensores del dispositivo como GPS, acelerómetros y giroscopios para determinar la posición y orientación del usuario en el espacio.
	Al analizar el entorno físico del usuario, a menudo con algoritmos y visión artificial, estos sistemas de realidad aumentada determinan dónde colocar el contenido digital.
	Lo que permite una experiencia más espontánea y dinámica.