\section{Realidad aumentada}
La realidad aumentada se refiere a la integración en tiempo real de información díggital
en el entorno del usuario. La tecnología de realidad aumentada superpone elementos virtuales,
enrequieciendo la percepción del mundo real. Esta tecnología se utiliza en diversas aplicaciones,
desde juegos y entretenimiento hasta educación, medicina y comercio. 
La realidad aumentada puede implementarse a través de dispositivos como smartphones, tabletas, gafas inteligentes y cascos de realidad aumentada, 
permitiendo a los usuarios interactuar con el contenido digital de manera intuitiva y enriquecedora.
\subsection {Evolución de la realidad aumentada} 
La realidad aumentada ha evolucionado significativamente desde sus inicios en la década de 1960.
Inicialmente, la tecnología era limitada y costosa, utilizada principalmente en entornos militares y de investigación.
Con el avance de la tecnología informática y la miniaturización de los dispositivos, la realidad aumentada se ha vuelto más accesible y versátil.
En la década de 1990, se desarrollaron las primeras aplicaciones comerciales de realidad aumentada, aunque su adopción fue limitada debido a las restricciones tecnológicas de la época.
En la década de 2000, la popularización de los smartphones impulsó su adopción masiva. Hoy en día, la realidad aumentada se integra en diversas industrias,
desde la educación hasta el comercio, y continúa evolucionando con avances en inteligencia artificial y otras tecnologías emergentes.
\subsection {Elementos de la realidad aumentada}
Los elementos clave de un sistema de realidad aumentada incluyen:
\begin{itemize}
	\item \textbf{Dispositivo de visualización:} Es el medio a través del cual el usuario observa la combinación del entorno real con los elementos virtuales.
	Puede ser una pantalla de teléfono móvil, tableta, gafas inteligentes o un casco de realidad aumentada.
	\item \textbf{Cámara:} Captura el entorno real en tiempo real, permitiendo que el sistema reconozca el espacio físico y posicione correctamente los objetos virtuales sobre él.
	\item \textbf{Hardware:} Conjunto de componentes físicos del dispositivo (procesador, sensores, GPU, batería, etc.) que permiten ejecutar las aplicaciones de realidad aumentada y procesar la información visual y espacial.
	\item \textbf{Software:} Es el conjunto de programas y algoritmos que interpretan los datos capturados por la cámara y los sensores, integrando los elementos virtuales en el entorno real de manera coherente y dinámica.
	\item \textbf{Marcador:} Es una imagen, patrón o código (como un código QR) que actúa como punto de referencia para que el software reconozca una posición específica y proyecte sobre ella el contenido digital.
\end{itemize}
\subsection{Tipos de realidad aumentada}
	Hay dos tipos fundamentales de realidad aumentada: \\
	\begin{itemize}
	\item \textbf{Por marcadores.} \\
	Superpone contenido dígital a un activador físico (es decir, un marcador), que puede ser una imagen, un objeto o un código QR.
	Cuando la cámara del dispositivos detecta el marcador, el software de realidad aumentada reconoce el patrón y proyecta el contenido virtual en la posición del marcador.
	Este tipo de realidad aumentada es común en aplicaciones educativas y de entretenimiento, 
	donde los usuarios pueden interactuar con modelos 3D, animaciones o información adicional al escanear un marcador específico.
	Dado que se pueda acceder a este tipo de realidad aumentada en cualquier momento y desde cualqiuer dispositivo,
	es el modelo de realidad aumentada más flexible y accesible. \\
	\item \textbf{Sin marcadores.} \\
	En contraparte, no requiere de un activador físico. Este tipo se basa en sensores del dispositivo como GPS, acelerómetros y giroscopios para determinar la posición y orientación del usuario en el espacio.
	Al analizar el entorno físico del usuario, a menudo con algoritmos y visión artificial, estos sistemas de realidad aumentada determinan dónde colocar el contenido digital.
	Lo que permite una experiencia más espontánea y dinámica.
	\end{itemize}
\subsection {Casos de uso}
\begin{itemize}
	\item \textbf{Educación:} 
	La realidad aumentada se utiliza para crear experiencias de aprendizaje interactivas, 
	permitiendo a los estudiantes explorar conceptos complejos a través de visualizaciones 3D y simulaciones.
	\item \textbf{Medicina:} 
	En el campo de la medicina, la realidad aumentada ayuda a los cirujanos a visualizar estructuras anatómicas durante procedimientos quirúrgicos, 
	mejorando la precisión y reduciendo riesgos.
	\item \textbf{Comercio:} 
	Las aplicaciones de realidad aumentada permiten a los clientes probar productos virtualmente antes de comprarlos, 
	como muebles o ropa, mejorando la experiencia de compra.
	\item \textbf{Juegos:} 
	Juegos utilizan la realidad aumentada para crear experiencias inmersivas que combinan el mundo real con elementos digitales.
	\item \textbf{Turismo:} 
	La realidad aumentada en aplicaciones turísticas proporciona información adicional sobre lugares históricos y culturales, 
	enriqueciendo la experiencia del visitante.
\end{itemize}
\input{}