\section{Actividades culturales y deportivas}
La UPIICSA fomenta la formación integral de sus estudiantes mediante la promoción de actividades culturales y deportivas.  
Estas actividades buscan fortalecer valores como el trabajo en equipo, la disciplina, la creatividad y la identidad politécnica, complementando la formación académica con espacios de desarrollo personal y social.

\subsection{Objetivo}
El principal objetivo de las actividades culturales y deportivas es contribuir al bienestar físico, emocional y social de la comunidad estudiantil, promoviendo hábitos saludables, la expresión artística y el sentido de pertenencia institucional.

\subsection{Actividades culturales}
La unidad cuenta con diversos talleres y grupos representativos que impulsan la creatividad, la apreciación artística y la difusión cultural.  
Los talleres están abiertos a estudiantes, docentes y personal administrativo, y se ofrecen tanto en nivel introductorio como avanzado.  

\begin{itemize}
	\item Danza folclórica mexicana  
	\item Teatro universitario  
	\item Coro institucional  
	\item Rondalla  
	\item Fotografía artística  
	\item Pintura y dibujo  
	\item Guitarra y canto  
\end{itemize}

Estos talleres permiten a los alumnos desarrollar habilidades expresivas y fortalecer su sensibilidad cultural, además de representar a la institución en festivales y concursos organizados por el Instituto Politécnico Nacional y otras universidades.

\subsection{Actividades deportivas}
La práctica deportiva en la UPIICSA busca fomentar un estilo de vida saludable y fortalecer la convivencia estudiantil.  
Los equipos representativos participan en torneos internos, regionales y nacionales, destacando en diversas disciplinas individuales y de conjunto.

\begin{itemize}
	\item Fútbol soccer  
	\item Fútbol rápido  
	\item Básquetbol  
	\item Voleibol  
	\item Atletismo  
	\item Tae Kwon Do  
	\item Natación  
	\item Ajedrez  
\end{itemize}

Además, se ofrecen programas recreativos y de acondicionamiento físico abiertos a toda la comunidad, promoviendo la actividad física como parte de la vida universitaria.

\subsection{Instalaciones}
UPIICSA cuenta con diversas instalaciones para el desarrollo de estas actividades:
\begin{itemize}
	\item Canchas de fútbol, básquetbol y voleibol.  
	\item Gimnasio techado para artes marciales y actividades físicas.  
	\item Salones de usos múltiples para danza, teatro y música.  
	\item Áreas al aire libre para entrenamiento y acondicionamiento físico.
\end{itemize}

\subsection{Importancia}
Las actividades culturales y deportivas fortalecen el sentido de comunidad y orgullo politécnico.  
Permiten a los estudiantes equilibrar su formación profesional con el desarrollo personal, emocional y artístico, promoviendo una educación integral acorde con los valores del Instituto Politécnico Nacional.

\subsection{Resumen de actividades}
A continuación, se muestra una tabla que resume las principales actividades culturales y deportivas que ofrece la UPIICSA.

\begin{table}[H]
	\centering
	\caption{Actividades culturales y deportivas en la UPIICSA}
	\label{tab:actividades_upiicsa}
	\begin{tabularx}{\textwidth}{|X|X|X|}
		\hline
		\textbf{Tipo de actividad} & \textbf{Ejemplos} & \textbf{Objetivo principal} \\
		\hline
		Culturales & Danza folclórica, teatro, coro, rondalla, fotografía, pintura, guitarra & Desarrollar la creatividad y fomentar la identidad cultural politécnica. \\
		\hline
		Deportivas & Fútbol, básquetbol, voleibol, atletismo, Tae Kwon Do, natación, ajedrez & Promover la salud física, la disciplina y el trabajo en equipo. \\
		\hline
		Recreativas & Acondicionamiento físico, torneos internos, actividades al aire libre & Fomentar la convivencia, la integración y el bienestar estudiantil. \\
		\hline
	\end{tabularx}
\end{table}
