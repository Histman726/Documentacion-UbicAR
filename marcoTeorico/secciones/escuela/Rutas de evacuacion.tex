\section{Rutas de evacuación en UPIICSA}

La Unidad Profesional Interdisciplinaria de Ingeniería y Ciencias Sociales y Administrativas (UPIICSA) cuenta con rutas de evacuación claramente señalizadas en cada uno de sus edificios, con el propósito de garantizar la seguridad de la comunidad estudiantil, docente y administrativa ante cualquier emergencia.

Las rutas de evacuación están diseñadas para guiar a las personas hacia zonas seguras y puntos de reunión previamente establecidos, minimizando el riesgo durante sismos, incendios u otras contingencias. 
Cada pasillo, aula y laboratorio dispone de señalizaciones fotoluminiscentes visibles aun en condiciones de poca luz.

\subsection{Zonas seguras y puntos de reunión}

Cada edificio de UPIICSA cuenta con áreas designadas como zonas seguras, a las cuales se debe dirigir la comunidad durante una evacuación:

\begin{itemize}
	\item \textbf{Edificio de Gobierno:} Las rutas conducen hacia el patio central y la explanada principal frente al edificio.
	\item \textbf{Laboratorios Pesados:} Los usuarios deben dirigirse hacia el área despejada contigua al edificio de Básicas.
	\item \textbf{Laboratorios Ligeros:} Las rutas guían hacia el estacionamiento posterior, que funge como zona de concentración.
	\item \textbf{Edificio de Básicas:} Las salidas principales llevan hacia el área verde lateral y hacia la explanada central.
	\item \textbf{Edificio de Ingeniería:} Los pasillos principales conducen hacia la plaza central frente a los edificios de Gobierno y Culturales.
	\item \textbf{Edificio de Culturales:} La evacuación se realiza hacia el patio central y las zonas abiertas adyacentes a los talleres deportivos.
\end{itemize}

\subsection{Recomendaciones generales de evacuación}

Durante una situación de emergencia, el personal y los estudiantes deben seguir las siguientes indicaciones:

\begin{enumerate}
	\item Mantener la calma y seguir las instrucciones del personal de brigadas de Protección Civil.
	\item No correr, empujar ni gritar durante la evacuación.
	\item No regresar por objetos personales una vez iniciada la salida.
	\item Dirigirse a las zonas seguras más cercanas siguiendo las señales indicativas.
	\item Esperar en el punto de reunión hasta recibir instrucciones de las autoridades correspondientes.
\end{enumerate}

El cumplimiento de estas medidas permite una evacuación ordenada y segura, contribuyendo a la protección de toda la comunidad politécnica.
