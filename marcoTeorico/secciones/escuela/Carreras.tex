\subsection{Carreras}
La Unidad Profesional Interdisciplinaria de Ingeniería y Ciencias Sociales y Administrativas (UPIICSA) ofrece programas académicos en las áreas de ingeniería, ciencias administrativas y sociales. Estas carreras se orientan a formar profesionales capaces de integrar el conocimiento técnico con una visión organizacional, fomentando la innovación y la solución de problemas reales en distintos sectores productivos.

\subsubsection{Carreras del área de Ingeniería}

\subsubsection{Ingeniería en Informática}
Forma profesionistas capaces de diseñar, desarrollar e implementar sistemas de información y soluciones tecnológicas para la gestión de datos y procesos empresariales.
\begin{itemize}
	\item \textbf{Duración:} 9 semestres  
	\item \textbf{Enfoque:} Desarrollo de software, bases de datos, redes y seguridad informática.  
	\item \textbf{Salidas profesionales:} Analista de sistemas, desarrollador de software, administrador de TI.
\end{itemize}

\subsubsection{Ingeniería Industrial}
Orienta al estudiante en la optimización de recursos humanos, materiales y tecnológicos dentro de las organizaciones, promoviendo la productividad y la eficiencia.
\begin{itemize}
	\item \textbf{Duración:} 9 semestres  
	\item \textbf{Enfoque:} Producción, calidad, logística y mejora continua.  
	\item \textbf{Salidas profesionales:} Ingeniero de procesos, consultor de calidad, gerente de producción.
\end{itemize}

\subsubsection{Ingeniería en Transporte}
Forma profesionales especializados en la planeación, operación y gestión de sistemas de transporte de personas y mercancías, considerando aspectos técnicos, económicos y ambientales.
\begin{itemize}
	\item \textbf{Duración:} 9 semestres  
	\item \textbf{Enfoque:} Logística, diseño de rutas, seguridad vial y movilidad sostenible.  
	\item \textbf{Salidas profesionales:} Planificador de transporte, gestor logístico, analista de movilidad.
\end{itemize}

\subsubsection{Ingeniería en Sistemas Automotrices}
Prepara ingenieros con conocimientos en el diseño, mantenimiento y diagnóstico de sistemas automotrices modernos, integrando mecánica, electrónica y control.
\begin{itemize}
	\item \textbf{Duración:} 9 semestres  
	\item \textbf{Enfoque:} Tecnología vehicular, sistemas eléctricos, motores y manufactura.  
	\item \textbf{Salidas profesionales:} Ingeniero automotriz, técnico en diagnóstico vehicular, desarrollador de sistemas mecatrónicos.
\end{itemize}

\subsubsection{Carreras del área de Ciencias Sociales y Administrativas}

\subsubsection{Licenciatura en Administración Industrial}
Forma profesionales con la capacidad de gestionar procesos administrativos, financieros y operativos dentro de las organizaciones, integrando herramientas tecnológicas y de gestión.
\begin{itemize}
	\item \textbf{Duración:} 8 semestres  
	\item \textbf{Enfoque:} Planeación estratégica, recursos humanos, finanzas y producción.  
	\item \textbf{Salidas profesionales:} Administrador de operaciones, jefe de área, analista de negocios.
\end{itemize}

\subsubsection{Licenciatura en Ciencias de la Informática}
Prepara profesionistas con una sólida formación en el desarrollo, análisis y gestión de sistemas informáticos aplicados a los procesos administrativos y de decisión organizacional.
\begin{itemize}
	\item \textbf{Duración:} 9 semestres  
	\item \textbf{Enfoque:} Programación, análisis de datos, inteligencia de negocios y gestión tecnológica.  
	\item \textbf{Salidas profesionales:} Analista de datos, desarrollador de software empresarial, consultor en TI.
\end{itemize}

\subsubsection{Licenciatura en Relaciones Comerciales}
Capacita profesionales especializados en mercadotecnia, comercio internacional y ventas, con una visión integral de los mercados nacionales y globales.
\begin{itemize}
	\item \textbf{Duración:} 8 semestres  
	\item \textbf{Enfoque:} Mercadotecnia, negocios internacionales, publicidad y gestión comercial.  
	\item \textbf{Salidas profesionales:} Ejecutivo de ventas, especialista en marketing, analista de comercio exterior.
\end{itemize}
\subsection{Programas de Posgrado}

\subsubsection{Maestría en Ciencias en Administración Industrial}
Dirigida a formar especialistas en gestión estratégica, innovación organizacional y liderazgo empresarial, con base en métodos científicos y tecnológicos modernos.
\begin{itemize}
	\item \textbf{Duración:} 4 semestres  
	\item \textbf{Enfoque:} Planeación estratégica, finanzas corporativas, optimización de recursos y gestión del cambio.  
	\item \textbf{Dirigido a:} Profesionales del área administrativa o de ingeniería interesados en la gestión industrial avanzada.
\end{itemize}

\subsubsection{Maestría en Ciencias en Ingeniería Industrial}
Prepara a los estudiantes en la investigación aplicada para la optimización de procesos productivos, calidad y eficiencia en sistemas industriales complejos.
\begin{itemize}
	\item \textbf{Duración:} 4 semestres  
	\item \textbf{Enfoque:} Modelado de procesos, mejora continua, estadística aplicada y sustentabilidad.  
	\item \textbf{Dirigido a:} Ingenieros y profesionistas con experiencia en manufactura o gestión de calidad.
\end{itemize}

\subsubsection{Maestría en Ciencias en Ingeniería de Sistemas}
Enfocada en la aplicación de métodos matemáticos, modelos de simulación y técnicas computacionales para la resolución de problemas organizacionales y tecnológicos.
\begin{itemize}
	\item \textbf{Duración:} 4 semestres  
	\item \textbf{Enfoque:} Modelado de sistemas, optimización, simulación y análisis de datos.  
	\item \textbf{Dirigido a:} Profesionales en informática, ingeniería o ciencias aplicadas.
\end{itemize}

\subsubsection{Doctorado en Ciencias en Ingeniería Industrial}
Orienta la formación hacia la investigación avanzada en optimización de procesos, sustentabilidad, innovación tecnológica y desarrollo industrial.
\begin{itemize}
	\item \textbf{Duración:} 6 a 8 semestres  
	\item \textbf{Enfoque:} Investigación científica aplicada a la ingeniería de procesos, productividad y competitividad industrial.  
	\item \textbf{Dirigido a:} Egresados de maestrías en ingeniería, ciencias o administración con interés en la investigación y docencia.
\end{itemize}

\subsubsection{Doctorado en Ciencias Administrativas}
Busca formar investigadores capaces de generar conocimiento original en el campo de la administración, la economía y la innovación empresarial.
\begin{itemize}
	\item \textbf{Duración:} 6 a 8 semestres  
	\item \textbf{Enfoque:} Teorías organizacionales, desarrollo sustentable, economía aplicada y políticas empresariales.  
	\item \textbf{Dirigido a:} Egresados de maestrías en administración, economía o áreas afines.
\end{itemize}

\subsection{Resumen general}
A continuación se presenta una tabla con las carreras, nivel académico y duración promedio de cada programa que ofrece la UPIICSA.

\begin{table}[H]
	\centering
	\caption{Carreras y posgrados que imparte la UPIICSA}
	\label{tab:carreras_upiicsa}
	\begin{tabularx}{\textwidth}{|X|X|X|}
		\hline
		\textbf{Programa académico} & \textbf{Nivel} & \textbf{Duración} \\
		\hline
		Ingeniería en Informática & Licenciatura & 9 semestres \\
		\hline
		Ingeniería Industrial & Licenciatura & 9 semestres \\
		\hline
		Ingeniería en Transporte & Licenciatura & 9 semestres \\
		\hline
		Ingeniería en Sistemas Automotrices & Licenciatura & 9 semestres \\
		\hline
		Licenciatura en Administración Industrial & Licenciatura & 8 semestres \\
		\hline
		Licenciatura en Ciencias de la Informática & Licenciatura & 9 semestres \\
		\hline
		Licenciatura en Relaciones Comerciales & Licenciatura & 8 semestres \\
		\hline
		Maestría en Ciencias en Administración Industrial & Maestría & 4 semestres \\
		\hline
		Maestría en Ciencias en Ingeniería Industrial & Maestría & 4 semestres \\
		\hline
		Maestría en Ciencias en Ingeniería de Sistemas & Maestría & 4 semestres \\
		\hline
		Doctorado en Ciencias en Ingeniería Industrial & Doctorado & 6--8 semestres \\
		\hline
		Doctorado en Ciencias Administrativas & Doctorado & 6--8 semestres \\
		\hline
	\end{tabularx}
\end{table}
