\section{CELEX UPIICSA}
El Centro de Lenguas Extranjeras (CELEX) de la UPIICSA es una dependencia académica del Instituto Politécnico Nacional dedicada a la enseñanza de idiomas extranjeros, con el propósito de fortalecer las competencias comunicativas de los estudiantes, egresados y del público en general.  
Su objetivo principal es contribuir al desarrollo académico y profesional de la comunidad mediante la formación lingüística y cultural.

\subsection{Objetivo}
El CELEX busca proporcionar a los participantes las herramientas lingüísticas necesarias para comunicarse eficazmente en contextos académicos, laborales y sociales.  
Asimismo, fomenta la comprensión intercultural y la competencia comunicativa en distintos idiomas, promoviendo una educación integral que complemente la formación profesional.

\subsection{Idiomas que ofrece}
El CELEX UPIICSA imparte cursos en distintos idiomas, adaptados a las necesidades y niveles de los estudiantes:
\begin{itemize}
	\item Inglés  
	\item Francés  
	\item Alemán  
	\item Italiano  
	\item Japonés  
\end{itemize}

\subsection{Estructura de niveles}
Cada idioma se organiza en módulos que cubren las principales habilidades lingüísticas: comprensión auditiva, expresión oral, comprensión lectora y expresión escrita.  
El CELEX maneja una estructura basada en el \textbf{Marco Común Europeo de Referencia para las Lenguas (MCER)}, la cual comprende los siguientes niveles:
\begin{itemize}
	\item A1 – Principiante  
	\item A2 – Básico  
	\item B1 – Intermedio  
	\item B2 – Intermedio alto  
	\item C1 – Avanzado  
	\item C2 – Dominio del idioma
\end{itemize}

\subsection{Modalidades de estudio}
El CELEX ofrece distintas modalidades para adaptarse a las necesidades de los estudiantes:
\begin{itemize}
	\item \textbf{Regular:} Clases presenciales con duración semestral.  
	\item \textbf{Intensivo:} Cursos acelerados con mayor carga horaria semanal.  
	\item \textbf{Sabatino:} Clases los fines de semana para estudiantes y trabajadores.  
	\item \textbf{En línea:} Modalidad virtual mediante plataformas educativas.
\end{itemize}

\subsection{Certificaciones y acreditaciones}
El CELEX prepara a los estudiantes para certificaciones internacionales reconocidas, tales como:
\begin{itemize}
	\item \textbf{Inglés:} TOEFL, IELTS, Cambridge.  
	\item \textbf{Francés:} DELF, DALF.  
	\item \textbf{Alemán:} Goethe-Zertifikat.  
	\item \textbf{Italiano:} CELI, CILS.  
	\item \textbf{Japonés:} JLPT.
\end{itemize}

\subsection{Importancia dentro de la UPIICSA}
El CELEX UPIICSA complementa la formación profesional de los estudiantes al proporcionarles una ventaja competitiva en el ámbito laboral y académico.  
El dominio de un idioma extranjero amplía las oportunidades de movilidad internacional, intercambio académico y participación en proyectos de investigación globales, fortaleciendo el perfil integral del egresado politécnico.

\begin{table}[H]
	\centering
	\caption{Idiomas, niveles y certificaciones ofrecidos por el CELEX UPIICSA}
	\label{tab:celex_upiicsa}
	\begin{tabularx}{\textwidth}{|X|X|X|}
		\hline
		\textbf{Idioma} & \textbf{Niveles (MCER)} & \textbf{Certificaciones disponibles} \\
		\hline
		Inglés & A1, A2, B1, B2, C1, C2 & TOEFL, IELTS, Cambridge (KET, PET, FCE, CAE, CPE) \\
		\hline
		Francés & A1, A2, B1, B2, C1, C2 & DELF, DALF \\
		\hline
		Alemán & A1, A2, B1, B2, C1 & Goethe-Zertifikat \\
		\hline
		Italiano & A1, A2, B1, B2, C1 & CELI, CILS \\
		\hline
		Japonés & N5, N4, N3, N2, N1 (equivalentes a niveles MCER) & JLPT (Japanese-Language Proficiency Test) \\
		\hline
	\end{tabularx}
\end{table}
