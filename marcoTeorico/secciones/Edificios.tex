\subsection{Edificios}
UPIICSA cuenta con varios edificios principales, cada uno con una función específica. A continuación, se describen sus características generales:

\subsubsection{Gobierno}
El edificio de Gobierno alberga las oficinas administrativas, dirección, subdirecciones y servicios escolares.  
\begin{itemize}
	\item \textbf{Número de pisos:} 2
	\item \textbf{Distribución:}  
	\begin{itemize}
		\item \textbf{Planta baja:} Dirección, Subdirección Académica, Control Escolar.  
		\item \textbf{Primer piso:} Recursos Humanos y áreas administrativas.  
	\end{itemize}
\end{itemize}

\subsubsection{Laboratorios pesados}
Este edificio está destinado a las prácticas de las áreas de ingeniería, especialmente aquellas que requieren equipos de gran tamaño o consumo energético elevado.  
\begin{itemize}
	\item \textbf{Número de pisos:} 2  
	\item \textbf{Distribución:}  
	\begin{itemize}
		\item \textbf{Planta baja:} Laboratorios de mecánica, electricidad y electrónica industrial.  
		\item \textbf{Primer piso:} Áreas de mantenimiento, instrumentación y control.  
	\end{itemize}
\end{itemize}

\subsubsection{Laboratorios ligeros}
Se utilizan para prácticas de cómputo, química básica y simulaciones.  
\begin{itemize}
	\item \textbf{Número de pisos:} 3  
	\item \textbf{Distribución:}  
	\begin{itemize}
		\item \textbf{Planta baja:} Laboratorios de informática y redes.  
		\item \textbf{Primer piso:} Laboratorios de química y física general.  
		\item \textbf{Segundo piso:} Laboratorios de simulación y desarrollo de software.  
	\end{itemize}
\end{itemize}

\subsubsection{Básicas}
En este edificio se imparten materias de tronco común como matemáticas, física y química.  
\begin{itemize}
	\item \textbf{Número de pisos:} 4  
	\item \textbf{Salones por piso:} Aproximadamente 10 aulas por nivel.  
	\item \textbf{Distribución:}  
	\begin{itemize}
		\item \textbf{Planta baja:} Humanísticas
		\item \textbf{Primer paso:} Matemáticas
		\item \textbf{Segundo piso:} Física  
		\item \textbf{Tercer piso:} Química
	\end{itemize}
\end{itemize}

\subsubsection{Ingeniería}
Dedicado principalmente a las materias de especialidad de las carreras de ingeniería.  
\begin{itemize}
	\item \textbf{Número de pisos:} 4  
	\item \textbf{Salones por piso:} 8 a 12 aulas.  
	\item \textbf{Distribución:}  
	\begin{itemize}
		\item \textbf{Planta baja:} Aulas de ingeniería industrial y mecánica.  
		\item \textbf{Primer piso:} Ingeniería en sistemas computacionales.  
		\item \textbf{Segundo piso:} Ingeniería en transporte.  
		\item \textbf{Tercer piso:} Laboratorios especializados y salas de proyectos.   
	\end{itemize}
\end{itemize}

\subsubsection{Culturales}
El edificio cultural alberga auditorios, salas de exposición y espacios para actividades extracurriculares.  
\begin{itemize}
	\item \textbf{Número de pisos:} 2  
	\item \textbf{Distribución:}  
	\begin{itemize}
		\item \textbf{Planta baja:} Auditorios, cafetería y galería de trofeos.  
		\item \textbf{Primer piso:} Salas de computo, salones de ingles y administrativos
	\end{itemize}
\end{itemize}