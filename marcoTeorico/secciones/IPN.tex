\section{Instituto Politécnico Nacional}
El \textbf{Instituto Politécnico Nacional (IPN)} es una de las instituciones educativas más importantes de México y América Latina, dedicada a la formación de profesionales en los campos científico, tecnológico y social.  
Fue fundado el 1º de enero de 1936 como resultado de una iniciativa del Gobierno Federal para consolidar la educación técnica y superior, bajo el lema \textit{"La Técnica al Servicio de la Patria"}.

\subsection{Historia y propósito}
El IPN nació con el objetivo de democratizar el acceso a la educación técnica y científica, ofreciendo oportunidades a los sectores sociales menos favorecidos.  
Su creación fue impulsada por figuras como Lázaro Cárdenas del Río, Juan de Dios Bátiz y Narciso Bassols, quienes visualizaron una institución que formara profesionales capaces de contribuir al desarrollo económico e industrial del país.

A lo largo de su historia, el Politécnico ha evolucionado para adaptarse a los retos contemporáneos, consolidándose como un pilar de la innovación tecnológica, la investigación aplicada y la vinculación con el sector productivo nacional e internacional.

\subsection{Misión}
Formar profesionales con una sólida preparación científica, tecnológica, humanística y ética que contribuyan al desarrollo económico, social y sostenible de México, mediante la investigación, la innovación y la difusión del conocimiento.

\subsection{Visión}
Ser una institución de educación superior reconocida a nivel nacional e internacional por la calidad de su enseñanza, investigación y vinculación, así como por su compromiso con la transformación social y el progreso tecnológico del país.

\subsection{Estructura académica}
El IPN está conformado por una amplia red de unidades académicas distribuidas en todo el territorio nacional, organizadas en tres niveles principales:
\begin{itemize}
	\item \textbf{Nivel medio superior:} Escuelas de nivel técnico y bachillerato como los Centros de Estudios Científicos y Tecnológicos (CECyT) y los Centros de Educación Media Superior (CET).
	\item \textbf{Nivel superior:} Unidades profesionales como las UPI (Unidades Profesionales Interdisciplinarias), las ESCOM, ESIME, ESIQIE, entre otras, donde se imparten licenciaturas e ingenierías.
	\item \textbf{Posgrado e investigación:} Centros de investigación como el CICATA, CINVESTAV y la SEPI, dedicados al desarrollo científico y tecnológico en múltiples disciplinas.
\end{itemize}

\subsection{Planteles de Nivel Superior}

A continuación, se listan las principales unidades académicas de nivel superior del Instituto Politécnico Nacional:

\begin{itemize}
	\item Escuela Superior de Ingeniería Mecánica y Eléctrica (ESIME), con sus unidades Zacatenco, Culhuacán, Ticomán y Azcapotzalco.
	\item Escuela Superior de Ingeniería Química e Industrias Extractivas (ESIQIE).
	\item Escuela Superior de Ingeniería y Arquitectura (ESIA), con sus unidades Zacatenco, Ticomán y Tecamachalco.
	\item Escuela Superior de Física y Matemáticas (ESFM).
	\item Escuela Superior de Cómputo (ESCOM).
	\item Escuela Superior de Comercio y Administración (ESCA), unidades Santo Tomás y Tepepan.
	\item Escuela Superior de Turismo (EST).
	\item Escuela Nacional de Ciencias Biológicas (ENCB).
	\item Escuela Nacional de Medicina y Homeopatía (ENMyH).
	\item Escuela Superior de Enfermería y Obstetricia (ESEO).
	\item Escuela Superior de Medicina (ESM).
	\item Escuela Superior de Economía (ESE).
	\item Escuela Superior de Ingeniería Textil (ESIT).
	\item Escuela Superior de Ingeniería y Textiles (ESIT).
	\item Escuela Superior de Ingeniería en Materiales (ESIMe).
	\item Unidad Profesional Interdisciplinaria de Ingeniería y Ciencias Sociales y Administrativas (UPIICSA).
	\item Unidad Profesional Interdisciplinaria de Biotecnología (UPIBI).
	\item Unidad Profesional Interdisciplinaria en Ingeniería y Tecnologías Avanzadas (UPIITA).
	\item Unidad Profesional Interdisciplinaria de Ingeniería Guanajuato (UPIIG).
	\item Unidad Profesional Interdisciplinaria de Ingeniería Hidalgo (UPIIH).
	\item Unidad Profesional Interdisciplinaria de Ingeniería Coahuila (UPIIC).
	\item Unidad Profesional Interdisciplinaria de Ingeniería Puebla (UPIIP).
	\item Escuela Superior de Ingeniería en Proyectos (ESIPro).
	\item Centro Interdisciplinario de Ciencias de la Salud (CICS), unidades Santo Tomás y Milpa Alta.
	\item Centro Interdisciplinario de Investigaciones y Estudios sobre Medio Ambiente y Desarrollo (CIIEMAD).
	\item Centro de Investigación en Ciencia Aplicada y Tecnología Avanzada (CICATA), con unidades en Legaria, Querétaro y Altamira.
	\item Centro de Innovación y Desarrollo Tecnológico en Cómputo (CIDETEC).
	\item Centro de Investigación e Innovación Tecnológica (CIITEC).
\end{itemize}

El IPN continúa expandiendo su presencia a nivel nacional mediante la apertura de nuevas unidades interdisciplinarias, fortaleciendo su compromiso con la educación pública, la investigación científica y la innovación tecnológica al servicio de México.

\subsection{Valores institucionales}
El Instituto Politécnico Nacional orienta su quehacer educativo bajo principios que fortalecen su identidad institucional:
\begin{itemize}
	\item Compromiso social
	\item Responsabilidad y ética profesional
	\item Innovación tecnológica
	\item Respeto y equidad
	\item Trabajo colaborativo
	\item Calidad educativa
\end{itemize}

\subsection{Reconocimiento y aportaciones}
El IPN ha sido reconocido por su destacada contribución al avance científico y tecnológico de México.  
Entre sus principales aportaciones se encuentran desarrollos en áreas de ingeniería, robótica, biotecnología, transporte, telecomunicaciones y salud.  
Además, mantiene una estrecha colaboración con empresas, gobiernos e instituciones educativas nacionales e internacionales para impulsar la transferencia de conocimiento y la formación de talento especializado.

\subsection{Símbolos institucionales}
\begin{itemize}
	\item \textbf{Lema:} \textit{“La Técnica al Servicio de la Patria”}.
	\item \textbf{Escudo:} Representa la fusión entre la ciencia y la técnica al servicio del desarrollo nacional.
	\item \textbf{Colores institucionales:} Guinda y blanco, símbolo de identidad, esfuerzo y pertenencia politécnica.
\end{itemize}
