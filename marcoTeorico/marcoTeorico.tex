\chapter{Marco teórico} 
En el presente capítulo se describen los conceptos teóricos y tecnológicos que sustentan el desarrollo del proyecto.
Se abordan temas relacionados con las herramientas y tecnologías utilizadas, así como los fundamentos de la realidad aumentada.

% --- Aquí agregas tus imágenes del bloque 2 ---
\begin{figure}[H]
    \centering
    \includegraphics[width=0.9\textwidth]{imagenes/tecnologias.png}
    \caption{Principales tecnologías empleadas en el desarrollo de UbicAR.}
    \label{fig:tecnologias}
\end{figure}

\begin{figure}[H]
    \centering
    \includegraphics[width=0.9\textwidth]{imagenes/flujo-localizacion.jpg}
    \caption{Flujo general del proceso de localización en el sistema UbicAR.}
    \label{fig:flujo-localizacion}
\end{figure}

\begin{figure}[H]
    \centering
    \includegraphics[width=0.9\textwidth]{imagenes/ejemplo-ar.png}
    \caption{Ejemplo de interfaz en realidad aumentada dentro del sistema UbicAR.}
    \label{fig:ejemplo-ar}
\end{figure}

% --- Luego continúan tus secciones ---

	\section{Unity}
Unity es una plataforma de desarrollo 3D en tiempo real para compilar aplicaciones 2D y 3D, 
como juegos y simulaciones, con .NET y el lenguaje de programación C#.
El impacto que tiene en el desarrollo de aplicaciones AR es significativo, 
ya que proporciona un entorno robusto y versátil para crear experiencias inmersivas y atractivas.
\subsection{Ventajas de usar Unity para AR}
Unity ofrece varias ventajas para el desarrollo de aplicaciones de realidad aumentada:
\begin{itemize}
    \item \textbf{Compatibilidad multiplataforma:} 
    Unity permite desarrollar aplicaciones AR que pueden ejecutarse en múltiples plataformas, como iOS, Android, entre otros, facilitando la distribución y el acceso a una audiencia más amplia.
    \item \textbf{Herramientas integradas:} 
    Unity proporciona herramientas específicas para AR, como AR Foundation, que simplifican la integración de funcionalidades de AR en las aplicaciones.
    \item \textbf{Comunidad activa:} 
    La amplia comunidad de desarrolladores de Unity ofrece recursos, tutoriales y soporte, lo que facilita la resolución de problemas y el aprendizaje continuo.
    \item \textbf{Gráficos avanzados:} 
    Unity ofrece capacidades gráficas avanzadas que permiten crear experiencias visualmente atractivas y realistas en aplicaciones AR.
    \item \textbf{Facilidad de uso:} 
    La interfaz intuitiva de Unity y su sistema de arrastrar y soltar facilitan el desarrollo rápido de prototipos y la iteración en el diseño de aplicaciones AR.
\end{itemize}
\subsection{Limitaciones de Unity para AR}
A pesar de sus ventajas, Unity también presenta algunas limitaciones en el desarrollo de aplicaciones AR, en este caso en la navegación de interiores.
\begin{itemize}
    \item \textbf{Precisión en la localización:}
    Unity no tiene herramientas nativas de su entorno para la localización precisa en interiores, lo que puede ser un desafío para aplicaciones de navegación.
    \item \textbf{Curva de aprendizaje:} 
    Aunque Unity es accesible, los desarrolladores nuevos pueden enfrentar una curva de aprendizaje significativa, especialmente si no están familiarizados con C# o el desarrollo 3D.
    \item \textbf{Rendimiento:} 
    Las aplicaciones AR pueden ser exigentes en términos de rendimiento, y optimizar las aplicaciones en Unity para dispositivos móviles puede ser un desafío.
    \item \textbf{Consumo de bateria:} 
    Las aplicaciones AR desarrolladas en Unity pueden consumir una cantidad significativa de batería en dispositivos móviles, lo que puede afectar la experiencia del usuario.
    \item \textbf{Datos en tiempo real:}
    Unity no proporciona soluciones integradas para la gestión de datos en tiempo real, 
    lo que puede ser necesario para aplicaciones de navegación en interiores que requieren actualizaciones constantes.
\end{itemize}

	\section{Vuforia}
Vuforia AR es una plataforma de desarrollo de realidad aumentada que permite crear aplicaciones capaces de reconocer imágenes, objetos, superficies y entornos del mundo real para superponer sobre ellos elementos digitales como modelos 3D, videos, texto o animaciones. Utiliza visión por computadora para rastrear con precisión los objetos y mantener el contenido virtual alineado con el entorno físico, lo que facilita experiencias interactivas en áreas como educación, industria, publicidad y entretenimiento.

\begin{figure}[H]
	\centering
	\includegraphics[width=0.60\textwidth]{Vuforia concepto.jpg}
	\caption{Plataforma de Vuforia para la Realidad Aumentada}
	\label{fig:concepto-Vuforia}
\end{figure}

\subsection{Funcionalidades}
\begin{itemize}
    \item \textbf{Image Targets:} 
    Vuforia puede reconocer y rastrear imágenes planas, conocidas como marcadores, para superponer contenido digital en el mundo real.
    \item \textbf{Cloud Image Recognition:} 
    Reconocer un gran conjunto de imágenes y actualizar con frecuencia la base de datos con nuevas imágenes.
    \textbf{Bases de datos en la nube frente a dispositivos:} 
    Conozca las diferencias entre los dos tipos de bases de datos y seleccione la solución que mejor se adapte a sus necesidades.
    \textbf{Servicios web de Vuforia:} 
    Con la API de VWS, puede administrar estas grandes bases de datos de imágenes en la nube de manera eficiente y puede automatizar sus flujos de trabajo en su propio sistema de administración de contenido.
    \item \textbf{Model Target:} 
    Permite reconocer objetos por forma utilizando modelos 3D preexistentes. 
    Coloque contenido de realidad aumentada en una amplia variedad de artículos como equipos industriales, vehículos, juguetes y electrodomésticos.
    \item \textbf{Ground Plane:} 
    Permite colocar contenido en superficies horizontales del entorno, como mesas y suelos.
    \item \textbf{VuMarks:} 
    Estos son marcadores personalizados que pueden codificar una variedad de formatos de datos. Admiten tanto la identificación única como el seguimiento de aplicaciones de RA.
    \item \textbf{Integración con Unity:} 
    Vuforia se integra fácilmente con Unity, lo que facilita el desarrollo de aplicaciones AR utilizando las herramientas y capacidades de Unity.
    \item \textbf{Cámara externa:}
    Accede a los datos de video desde una cámara fuera de la de un teléfono o tableta al crear experiencias de realidad aumentada. 
    La cámara externa se utiliza en extensión del marco de controladores.
    \item \textbf{Vuforia Fusion:}
    Diseñado para proporcionar la mejor experiencia de RA posible en una amplia gama de dispositivos. 
    Fusion detecta las capacidades del dispositivo subyacente (como ARKit/ARCore) y las fusiona con las funciones de Vuforia Engine, 
    lo que permite a los desarrolladores confiar en una única API de Vuforia para una experiencia de realidad aumentada óptima.
    \item \textbf{Grabación  y reproducción:}
    Grabe y reproduzca su sesión de RA para probar, experimentar y mejorar su flujo de trabajo de desarrollo de RA con la grabadora de sesiones. 
    Use la API o el GameObject SessionRecorder listo para usar en Unity para registrar sus destinos de Vuforia y continuar desarrollando incluso cuando esos destinos o espacios no estén disponibles.
\end{itemize}

\subsection{Comparación con otros SDKs}
\begin{itemize}
    \item \textbf{ARKit y ARCore:} 
    Mientras que ARKit (iOS) y ARCore (Android) se centran en el reconocimiento de superficies y la detección de planos, Vuforia destaca en el reconocimiento de imágenes y objetos, ofreciendo una mayor variedad de opciones para desarrolladores.
    \item \textbf{Wikitude:} 
    Wikitude es otro SDK popular que ofrece funcionalidades similares a Vuforia. Sin embargo, Vuforia es conocido por su facilidad de uso y su integración con Unity, lo que lo hace más accesible para desarrolladores principiantes.
    \item \textbf{EasyAR:} 
    EasyAR es una alternativa más económica a Vuforia, pero puede carecer de algunas funcionalidades avanzadas que Vuforia ofrece, como el soporte para Model Targets y VuMarks.
    \item \textbf{Kudan:}
    Kudan es otro SDK de RA que se centra en el reconocimiento de imágenes y el seguimiento. Sin embargo, Vuforia ofrece una gama más amplia de funcionalidades y una comunidad de desarrolladores más grande.
\end{itemize}
	\section{Realidad aumentada}
La realidad aumentada se refiere a la integración en tiempo real de información díggital
en el entorno del usuario. La tecnología de realidad aumentada superpone elementos virtuales,
enrequieciendo la percepción del mundo real. Esta tecnología se utiliza en diversas aplicaciones,
desde juegos y entretenimiento hasta educación, medicina y comercio. 
La realidad aumentada puede implementarse a través de dispositivos como smartphones, tabletas, gafas inteligentes y cascos de realidad aumentada, 
permitiendo a los usuarios interactuar con el contenido digital de manera intuitiva y enriquecedora.
\subsection {Evolución de la realidad aumentada} 
\subsection {Componentes de la realidad aumentada}
\subsection{Tipos de realidad aumentada}
	Hay dos tipos fundamentales de realidad aumentada: \\
	\textbf {Por marcadores.} \\
	Superpone contenido dígital a un activador físico (es decir, un marcador), que puede ser una imagen, un objeto o un código QR.
	Cuando la cámara del dispositivos detecta el marcador, el software de realidad aumentada reconoce el patrón y proyecta el contenido virtual en la posición del marcador.
	Este tipo de realidad aumentada es común en aplicaciones educativas y de entretenimiento, 
	donde los usuarios pueden interactuar con modelos 3D, animaciones o información adicional al escanear un marcador específico.
	Dado que se pueda acceder a este tipo de realidad aumentada en cualquier momento y desde cualqiuer dispositivo,
	es el modelo de realidad aumentada más flexible y accesible. \\
	\textbf {Sin marcadores.} \\
	En contraparte, no requiere de un activador físico. Este tipo se basa en sensores del dispositivo como GPS, acelerómetros y giroscopios para determinar la posición y orientación del usuario en el espacio.
	Al analizar el entorno físico del usuario, a menudo con algoritmos y visión artificial, estos sistemas de realidad aumentada determinan dónde colocar el contenido digital.
	Lo que permite una experiencia más espontánea y dinámica.
	\section{Instituto Politécnico Nacional}
El \textbf{Instituto Politécnico Nacional (IPN)} es una de las instituciones educativas más importantes de México y América Latina, dedicada a la formación de profesionales en los campos científico, tecnológico y social.  
Fue fundado el 1º de enero de 1936 como resultado de una iniciativa del Gobierno Federal para consolidar la educación técnica y superior, bajo el lema \textit{"La Técnica al Servicio de la Patria"}.

\subsection{Historia y propósito}
El IPN nació con el objetivo de democratizar el acceso a la educación técnica y científica, ofreciendo oportunidades a los sectores sociales menos favorecidos.  
Su creación fue impulsada por figuras como Lázaro Cárdenas del Río, Juan de Dios Bátiz y Narciso Bassols, quienes visualizaron una institución que formara profesionales capaces de contribuir al desarrollo económico e industrial del país.

A lo largo de su historia, el Politécnico ha evolucionado para adaptarse a los retos contemporáneos, consolidándose como un pilar de la innovación tecnológica, la investigación aplicada y la vinculación con el sector productivo nacional e internacional.

\subsection{Misión}
Formar profesionales con una sólida preparación científica, tecnológica, humanística y ética que contribuyan al desarrollo económico, social y sostenible de México, mediante la investigación, la innovación y la difusión del conocimiento.

\subsection{Visión}
Ser una institución de educación superior reconocida a nivel nacional e internacional por la calidad de su enseñanza, investigación y vinculación, así como por su compromiso con la transformación social y el progreso tecnológico del país.

\subsection{Estructura académica}
El IPN está conformado por una amplia red de unidades académicas distribuidas en todo el territorio nacional, organizadas en tres niveles principales:
\begin{itemize}
	\item \textbf{Nivel medio superior:} Escuelas de nivel técnico y bachillerato como los Centros de Estudios Científicos y Tecnológicos (CECyT) y los Centros de Educación Media Superior (CET).
	\item \textbf{Nivel superior:} Unidades profesionales como las UPI (Unidades Profesionales Interdisciplinarias), las ESCOM, ESIME, ESIQIE, entre otras, donde se imparten licenciaturas e ingenierías.
	\item \textbf{Posgrado e investigación:} Centros de investigación como el CICATA, CINVESTAV y la SEPI, dedicados al desarrollo científico y tecnológico en múltiples disciplinas.
\end{itemize}

\subsection{Planteles de Nivel Superior}

A continuación, se listan las principales unidades académicas de nivel superior del Instituto Politécnico Nacional:

\begin{itemize}
	\item Escuela Superior de Ingeniería Mecánica y Eléctrica (ESIME), con sus unidades Zacatenco, Culhuacán, Ticomán y Azcapotzalco.
	\item Escuela Superior de Ingeniería Química e Industrias Extractivas (ESIQIE).
	\item Escuela Superior de Ingeniería y Arquitectura (ESIA), con sus unidades Zacatenco, Ticomán y Tecamachalco.
	\item Escuela Superior de Física y Matemáticas (ESFM).
	\item Escuela Superior de Cómputo (ESCOM).
	\item Escuela Superior de Comercio y Administración (ESCA), unidades Santo Tomás y Tepepan.
	\item Escuela Superior de Turismo (EST).
	\item Escuela Nacional de Ciencias Biológicas (ENCB).
	\item Escuela Nacional de Medicina y Homeopatía (ENMyH).
	\item Escuela Superior de Enfermería y Obstetricia (ESEO).
	\item Escuela Superior de Medicina (ESM).
	\item Escuela Superior de Economía (ESE).
	\item Escuela Superior de Ingeniería Textil (ESIT).
	\item Escuela Superior de Ingeniería y Textiles (ESIT).
	\item Escuela Superior de Ingeniería en Materiales (ESIMe).
	\item Unidad Profesional Interdisciplinaria de Ingeniería y Ciencias Sociales y Administrativas (UPIICSA).
	\item Unidad Profesional Interdisciplinaria de Biotecnología (UPIBI).
	\item Unidad Profesional Interdisciplinaria en Ingeniería y Tecnologías Avanzadas (UPIITA).
	\item Unidad Profesional Interdisciplinaria de Ingeniería Guanajuato (UPIIG).
	\item Unidad Profesional Interdisciplinaria de Ingeniería Hidalgo (UPIIH).
	\item Unidad Profesional Interdisciplinaria de Ingeniería Coahuila (UPIIC).
	\item Unidad Profesional Interdisciplinaria de Ingeniería Puebla (UPIIP).
	\item Escuela Superior de Ingeniería en Proyectos (ESIPro).
	\item Centro Interdisciplinario de Ciencias de la Salud (CICS), unidades Santo Tomás y Milpa Alta.
	\item Centro Interdisciplinario de Investigaciones y Estudios sobre Medio Ambiente y Desarrollo (CIIEMAD).
	\item Centro de Investigación en Ciencia Aplicada y Tecnología Avanzada (CICATA), con unidades en Legaria, Querétaro y Altamira.
	\item Centro de Innovación y Desarrollo Tecnológico en Cómputo (CIDETEC).
	\item Centro de Investigación e Innovación Tecnológica (CIITEC).
\end{itemize}

El IPN continúa expandiendo su presencia a nivel nacional mediante la apertura de nuevas unidades interdisciplinarias, fortaleciendo su compromiso con la educación pública, la investigación científica y la innovación tecnológica al servicio de México.

\subsection{Valores institucionales}
El Instituto Politécnico Nacional orienta su quehacer educativo bajo principios que fortalecen su identidad institucional:
\begin{itemize}
	\item Compromiso social
	\item Responsabilidad y ética profesional
	\item Innovación tecnológica
	\item Respeto y equidad
	\item Trabajo colaborativo
	\item Calidad educativa
\end{itemize}

\subsection{Reconocimiento y aportaciones}
El IPN ha sido reconocido por su destacada contribución al avance científico y tecnológico de México.  
Entre sus principales aportaciones se encuentran desarrollos en áreas de ingeniería, robótica, biotecnología, transporte, telecomunicaciones y salud.  
Además, mantiene una estrecha colaboración con empresas, gobiernos e instituciones educativas nacionales e internacionales para impulsar la transferencia de conocimiento y la formación de talento especializado.

\subsection{Símbolos institucionales}
\begin{itemize}
	\item \textbf{Lema:} \textit{“La Técnica al Servicio de la Patria”}.
	\item \textbf{Escudo:} Representa la fusión entre la ciencia y la técnica al servicio del desarrollo nacional.
	\item \textbf{Colores institucionales:} Guinda y blanco, símbolo de identidad, esfuerzo y pertenencia politécnica.
\end{itemize}

	\section{UPIICSA}
La Unidad Profesional Interdisciplinaria de Ingeniería y Ciencias Sociales y Administrativas (UPIICSA) es una de las escuelas del Instituto Politécnico Nacional (IPN), ubicada en la Ciudad de México. Su infraestructura está conformada por diversos edificios que albergan aulas, laboratorios, oficinas administrativas y espacios culturales, distribuidos estratégicamente para cubrir las necesidades académicas y de investigación de la comunidad estudiantil.

\subsection{Edificios}
UPIICSA cuenta con varios edificios principales, cada uno con una función específica. A continuación, se describen sus características generales:

\subsubsection{Gobierno}
El edificio de Gobierno alberga las oficinas administrativas, dirección, subdirecciones y servicios escolares.  
\begin{itemize}
	\item \textbf{Número de pisos:} 2
	\item \textbf{Distribución:}  
	\begin{itemize}
		\item \textbf{Planta baja:} Dirección, Subdirección Académica, Control Escolar.  
		\item \textbf{Primer piso:} Recursos Humanos y áreas administrativas.  
	\end{itemize}
\end{itemize}

\subsubsection{Laboratorios pesados}
Este edificio está destinado a las prácticas de las áreas de ingeniería, especialmente aquellas que requieren equipos de gran tamaño o consumo energético elevado.  
\begin{itemize}
	\item \textbf{Número de pisos:} 2  
	\item \textbf{Distribución:}  
	\begin{itemize}
		\item \textbf{Planta baja:} Laboratorios de mecánica, electricidad y electrónica industrial.  
		\item \textbf{Primer piso:} Áreas de mantenimiento, instrumentación y control.  
	\end{itemize}
\end{itemize}

\subsubsection{Laboratorios ligeros}
Se utilizan para prácticas de cómputo, química básica y simulaciones.  
\begin{itemize}
	\item \textbf{Número de pisos:} 3  
	\item \textbf{Distribución:}  
	\begin{itemize}
		\item \textbf{Planta baja:} Laboratorios de informática y redes.  
		\item \textbf{Primer piso:} Laboratorios de química y física general.  
		\item \textbf{Segundo piso:} Laboratorios de simulación y desarrollo de software.  
	\end{itemize}
\end{itemize}

\subsubsection{Básicas}
En este edificio se imparten materias de tronco común como matemáticas, física y química.  
\begin{itemize}
	\item \textbf{Número de pisos:} 4  
	\item \textbf{Salones por piso:} Aproximadamente 10 aulas por nivel.  
	\item \textbf{Distribución:}  
	\begin{itemize}
		\item \textbf{Planta baja:} Humanísticas
		\item \textbf{Primer paso:} Matemáticas
		\item \textbf{Segundo piso:} Física  
		\item \textbf{Tercer piso:} Química
	\end{itemize}
\end{itemize}

\subsubsection{Ingeniería}
Dedicado principalmente a las materias de especialidad de las carreras de ingeniería.  
\begin{itemize}
	\item \textbf{Número de pisos:} 4  
	\item \textbf{Salones por piso:} 8 a 12 aulas.  
	\item \textbf{Distribución:}  
	\begin{itemize}
		\item \textbf{Planta baja:} Aulas de ingeniería industrial y mecánica.  
		\item \textbf{Primer piso:} Ingeniería en sistemas computacionales.  
		\item \textbf{Segundo piso:} Ingeniería en transporte.  
		\item \textbf{Tercer piso:} Laboratorios especializados y salas de proyectos.   
	\end{itemize}
\end{itemize}

\subsubsection{Culturales}
El edificio cultural alberga auditorios, salas de exposición y espacios para actividades extracurriculares.  
\begin{itemize}
	\item \textbf{Número de pisos:} 2  
	\item \textbf{Distribución:}  
	\begin{itemize}
		\item \textbf{Planta baja:} Auditorios, cafetería y galería de trofeos.  
		\item \textbf{Primer piso:} Salas de computo, salones de ingles y administrativos
	\end{itemize}
\end{itemize}
\subsection{Carreras}
La Unidad Profesional Interdisciplinaria de Ingeniería y Ciencias Sociales y Administrativas (UPIICSA) ofrece programas académicos en las áreas de ingeniería, ciencias administrativas y sociales. Estas carreras se orientan a formar profesionales capaces de integrar el conocimiento técnico con una visión organizacional, fomentando la innovación y la solución de problemas reales en distintos sectores productivos.

\subsubsection{Carreras del área de Ingeniería}

\subsubsection{Ingeniería en Informática}
Forma profesionistas capaces de diseñar, desarrollar e implementar sistemas de información y soluciones tecnológicas para la gestión de datos y procesos empresariales.
\begin{itemize}
	\item \textbf{Duración:} 9 semestres  
	\item \textbf{Enfoque:} Desarrollo de software, bases de datos, redes y seguridad informática.  
	\item \textbf{Salidas profesionales:} Analista de sistemas, desarrollador de software, administrador de TI.
\end{itemize}

\subsubsection{Ingeniería Industrial}
Orienta al estudiante en la optimización de recursos humanos, materiales y tecnológicos dentro de las organizaciones, promoviendo la productividad y la eficiencia.
\begin{itemize}
	\item \textbf{Duración:} 9 semestres  
	\item \textbf{Enfoque:} Producción, calidad, logística y mejora continua.  
	\item \textbf{Salidas profesionales:} Ingeniero de procesos, consultor de calidad, gerente de producción.
\end{itemize}

\subsubsection{Ingeniería en Transporte}
Forma profesionales especializados en la planeación, operación y gestión de sistemas de transporte de personas y mercancías, considerando aspectos técnicos, económicos y ambientales.
\begin{itemize}
	\item \textbf{Duración:} 9 semestres  
	\item \textbf{Enfoque:} Logística, diseño de rutas, seguridad vial y movilidad sostenible.  
	\item \textbf{Salidas profesionales:} Planificador de transporte, gestor logístico, analista de movilidad.
\end{itemize}

\subsubsection{Ingeniería en Sistemas Automotrices}
Prepara ingenieros con conocimientos en el diseño, mantenimiento y diagnóstico de sistemas automotrices modernos, integrando mecánica, electrónica y control.
\begin{itemize}
	\item \textbf{Duración:} 9 semestres  
	\item \textbf{Enfoque:} Tecnología vehicular, sistemas eléctricos, motores y manufactura.  
	\item \textbf{Salidas profesionales:} Ingeniero automotriz, técnico en diagnóstico vehicular, desarrollador de sistemas mecatrónicos.
\end{itemize}

\subsubsection{Carreras del área de Ciencias Sociales y Administrativas}

\subsubsection{Licenciatura en Administración Industrial}
Forma profesionales con la capacidad de gestionar procesos administrativos, financieros y operativos dentro de las organizaciones, integrando herramientas tecnológicas y de gestión.
\begin{itemize}
	\item \textbf{Duración:} 8 semestres  
	\item \textbf{Enfoque:} Planeación estratégica, recursos humanos, finanzas y producción.  
	\item \textbf{Salidas profesionales:} Administrador de operaciones, jefe de área, analista de negocios.
\end{itemize}

\subsubsection{Licenciatura en Ciencias de la Informática}
Prepara profesionistas con una sólida formación en el desarrollo, análisis y gestión de sistemas informáticos aplicados a los procesos administrativos y de decisión organizacional.
\begin{itemize}
	\item \textbf{Duración:} 9 semestres  
	\item \textbf{Enfoque:} Programación, análisis de datos, inteligencia de negocios y gestión tecnológica.  
	\item \textbf{Salidas profesionales:} Analista de datos, desarrollador de software empresarial, consultor en TI.
\end{itemize}

\subsubsection{Licenciatura en Relaciones Comerciales}
Capacita profesionales especializados en mercadotecnia, comercio internacional y ventas, con una visión integral de los mercados nacionales y globales.
\begin{itemize}
	\item \textbf{Duración:} 8 semestres  
	\item \textbf{Enfoque:} Mercadotecnia, negocios internacionales, publicidad y gestión comercial.  
	\item \textbf{Salidas profesionales:} Ejecutivo de ventas, especialista en marketing, analista de comercio exterior.
\end{itemize}
\subsection{Programas de Posgrado}

\subsubsection{Maestría en Ciencias en Administración Industrial}
Dirigida a formar especialistas en gestión estratégica, innovación organizacional y liderazgo empresarial, con base en métodos científicos y tecnológicos modernos.
\begin{itemize}
	\item \textbf{Duración:} 4 semestres  
	\item \textbf{Enfoque:} Planeación estratégica, finanzas corporativas, optimización de recursos y gestión del cambio.  
	\item \textbf{Dirigido a:} Profesionales del área administrativa o de ingeniería interesados en la gestión industrial avanzada.
\end{itemize}

\subsubsection{Maestría en Ciencias en Ingeniería Industrial}
Prepara a los estudiantes en la investigación aplicada para la optimización de procesos productivos, calidad y eficiencia en sistemas industriales complejos.
\begin{itemize}
	\item \textbf{Duración:} 4 semestres  
	\item \textbf{Enfoque:} Modelado de procesos, mejora continua, estadística aplicada y sustentabilidad.  
	\item \textbf{Dirigido a:} Ingenieros y profesionistas con experiencia en manufactura o gestión de calidad.
\end{itemize}

\subsubsection{Maestría en Ciencias en Ingeniería de Sistemas}
Enfocada en la aplicación de métodos matemáticos, modelos de simulación y técnicas computacionales para la resolución de problemas organizacionales y tecnológicos.
\begin{itemize}
	\item \textbf{Duración:} 4 semestres  
	\item \textbf{Enfoque:} Modelado de sistemas, optimización, simulación y análisis de datos.  
	\item \textbf{Dirigido a:} Profesionales en informática, ingeniería o ciencias aplicadas.
\end{itemize}

\subsubsection{Doctorado en Ciencias en Ingeniería Industrial}
Orienta la formación hacia la investigación avanzada en optimización de procesos, sustentabilidad, innovación tecnológica y desarrollo industrial.
\begin{itemize}
	\item \textbf{Duración:} 6 a 8 semestres  
	\item \textbf{Enfoque:} Investigación científica aplicada a la ingeniería de procesos, productividad y competitividad industrial.  
	\item \textbf{Dirigido a:} Egresados de maestrías en ingeniería, ciencias o administración con interés en la investigación y docencia.
\end{itemize}

\subsubsection{Doctorado en Ciencias Administrativas}
Busca formar investigadores capaces de generar conocimiento original en el campo de la administración, la economía y la innovación empresarial.
\begin{itemize}
	\item \textbf{Duración:} 6 a 8 semestres  
	\item \textbf{Enfoque:} Teorías organizacionales, desarrollo sustentable, economía aplicada y políticas empresariales.  
	\item \textbf{Dirigido a:} Egresados de maestrías en administración, economía o áreas afines.
\end{itemize}

\subsection{Resumen general}
A continuación se presenta una tabla con las carreras, nivel académico y duración promedio de cada programa que ofrece la UPIICSA.

\begin{table}[H]
	\centering
	\caption{Carreras y posgrados que imparte la UPIICSA}
	\label{tab:carreras_upiicsa}
	\begin{tabularx}{\textwidth}{|X|X|X|}
		\hline
		\textbf{Programa académico} & \textbf{Nivel} & \textbf{Duración} \\
		\hline
		Ingeniería en Informática & Licenciatura & 9 semestres \\
		\hline
		Ingeniería Industrial & Licenciatura & 9 semestres \\
		\hline
		Ingeniería en Transporte & Licenciatura & 9 semestres \\
		\hline
		Ingeniería en Sistemas Automotrices & Licenciatura & 9 semestres \\
		\hline
		Licenciatura en Administración Industrial & Licenciatura & 8 semestres \\
		\hline
		Licenciatura en Ciencias de la Informática & Licenciatura & 9 semestres \\
		\hline
		Licenciatura en Relaciones Comerciales & Licenciatura & 8 semestres \\
		\hline
		Maestría en Ciencias en Administración Industrial & Maestría & 4 semestres \\
		\hline
		Maestría en Ciencias en Ingeniería Industrial & Maestría & 4 semestres \\
		\hline
		Maestría en Ciencias en Ingeniería de Sistemas & Maestría & 4 semestres \\
		\hline
		Doctorado en Ciencias en Ingeniería Industrial & Doctorado & 6--8 semestres \\
		\hline
		Doctorado en Ciencias Administrativas & Doctorado & 6--8 semestres \\
		\hline
	\end{tabularx}
\end{table}

\section{CELEX UPIICSA}
El Centro de Lenguas Extranjeras (CELEX) de la UPIICSA es una dependencia académica del Instituto Politécnico Nacional dedicada a la enseñanza de idiomas extranjeros, con el propósito de fortalecer las competencias comunicativas de los estudiantes, egresados y del público en general.  
Su objetivo principal es contribuir al desarrollo académico y profesional de la comunidad mediante la formación lingüística y cultural.

\subsection{Objetivo}
El CELEX busca proporcionar a los participantes las herramientas lingüísticas necesarias para comunicarse eficazmente en contextos académicos, laborales y sociales.  
Asimismo, fomenta la comprensión intercultural y la competencia comunicativa en distintos idiomas, promoviendo una educación integral que complemente la formación profesional.

\subsection{Idiomas que ofrece}
El CELEX UPIICSA imparte cursos en distintos idiomas, adaptados a las necesidades y niveles de los estudiantes:
\begin{itemize}
	\item Inglés  
	\item Francés  
	\item Alemán  
	\item Italiano  
	\item Japonés  
\end{itemize}

\subsection{Estructura de niveles}
Cada idioma se organiza en módulos que cubren las principales habilidades lingüísticas: comprensión auditiva, expresión oral, comprensión lectora y expresión escrita.  
El CELEX maneja una estructura basada en el \textbf{Marco Común Europeo de Referencia para las Lenguas (MCER)}, la cual comprende los siguientes niveles:
\begin{itemize}
	\item A1 – Principiante  
	\item A2 – Básico  
	\item B1 – Intermedio  
	\item B2 – Intermedio alto  
	\item C1 – Avanzado  
	\item C2 – Dominio del idioma
\end{itemize}

\subsection{Modalidades de estudio}
El CELEX ofrece distintas modalidades para adaptarse a las necesidades de los estudiantes:
\begin{itemize}
	\item \textbf{Regular:} Clases presenciales con duración semestral.  
	\item \textbf{Intensivo:} Cursos acelerados con mayor carga horaria semanal.  
	\item \textbf{Sabatino:} Clases los fines de semana para estudiantes y trabajadores.  
	\item \textbf{En línea:} Modalidad virtual mediante plataformas educativas.
\end{itemize}

\subsection{Certificaciones y acreditaciones}
El CELEX prepara a los estudiantes para certificaciones internacionales reconocidas, tales como:
\begin{itemize}
	\item \textbf{Inglés:} TOEFL, IELTS, Cambridge.  
	\item \textbf{Francés:} DELF, DALF.  
	\item \textbf{Alemán:} Goethe-Zertifikat.  
	\item \textbf{Italiano:} CELI, CILS.  
	\item \textbf{Japonés:} JLPT.
\end{itemize}

\subsection{Importancia dentro de la UPIICSA}
El CELEX UPIICSA complementa la formación profesional de los estudiantes al proporcionarles una ventaja competitiva en el ámbito laboral y académico.  
El dominio de un idioma extranjero amplía las oportunidades de movilidad internacional, intercambio académico y participación en proyectos de investigación globales, fortaleciendo el perfil integral del egresado politécnico.

\begin{table}[H]
	\centering
	\caption{Idiomas, niveles y certificaciones ofrecidos por el CELEX UPIICSA}
	\label{tab:celex_upiicsa}
	\begin{tabularx}{\textwidth}{|X|X|X|}
		\hline
		\textbf{Idioma} & \textbf{Niveles (MCER)} & \textbf{Certificaciones disponibles} \\
		\hline
		Inglés & A1, A2, B1, B2, C1, C2 & TOEFL, IELTS, Cambridge (KET, PET, FCE, CAE, CPE) \\
		\hline
		Francés & A1, A2, B1, B2, C1, C2 & DELF, DALF \\
		\hline
		Alemán & A1, A2, B1, B2, C1 & Goethe-Zertifikat \\
		\hline
		Italiano & A1, A2, B1, B2, C1 & CELI, CILS \\
		\hline
		Japonés & N5, N4, N3, N2, N1 (equivalentes a niveles MCER) & JLPT (Japanese-Language Proficiency Test) \\
		\hline
	\end{tabularx}
\end{table}

\section{Actividades culturales y deportivas}
La UPIICSA fomenta la formación integral de sus estudiantes mediante la promoción de actividades culturales y deportivas.  
Estas actividades buscan fortalecer valores como el trabajo en equipo, la disciplina, la creatividad y la identidad politécnica, complementando la formación académica con espacios de desarrollo personal y social.

\subsection{Objetivo}
El principal objetivo de las actividades culturales y deportivas es contribuir al bienestar físico, emocional y social de la comunidad estudiantil, promoviendo hábitos saludables, la expresión artística y el sentido de pertenencia institucional.

\subsection{Actividades culturales}
La unidad cuenta con diversos talleres y grupos representativos que impulsan la creatividad, la apreciación artística y la difusión cultural.  
Los talleres están abiertos a estudiantes, docentes y personal administrativo, y se ofrecen tanto en nivel introductorio como avanzado.  

\begin{itemize}
	\item Danza folclórica mexicana  
	\item Teatro universitario  
	\item Coro institucional  
	\item Rondalla  
	\item Fotografía artística  
	\item Pintura y dibujo  
	\item Guitarra y canto  
\end{itemize}

Estos talleres permiten a los alumnos desarrollar habilidades expresivas y fortalecer su sensibilidad cultural, además de representar a la institución en festivales y concursos organizados por el Instituto Politécnico Nacional y otras universidades.

\subsection{Actividades deportivas}
La práctica deportiva en la UPIICSA busca fomentar un estilo de vida saludable y fortalecer la convivencia estudiantil.  
Los equipos representativos participan en torneos internos, regionales y nacionales, destacando en diversas disciplinas individuales y de conjunto.

\begin{itemize}
	\item Fútbol soccer  
	\item Fútbol rápido  
	\item Básquetbol  
	\item Voleibol  
	\item Atletismo  
	\item Tae Kwon Do  
	\item Natación  
	\item Ajedrez  
\end{itemize}

Además, se ofrecen programas recreativos y de acondicionamiento físico abiertos a toda la comunidad, promoviendo la actividad física como parte de la vida universitaria.

\subsection{Instalaciones}
UPIICSA cuenta con diversas instalaciones para el desarrollo de estas actividades:
\begin{itemize}
	\item Canchas de fútbol, básquetbol y voleibol.  
	\item Gimnasio techado para artes marciales y actividades físicas.  
	\item Salones de usos múltiples para danza, teatro y música.  
	\item Áreas al aire libre para entrenamiento y acondicionamiento físico.
\end{itemize}

\subsection{Importancia}
Las actividades culturales y deportivas fortalecen el sentido de comunidad y orgullo politécnico.  
Permiten a los estudiantes equilibrar su formación profesional con el desarrollo personal, emocional y artístico, promoviendo una educación integral acorde con los valores del Instituto Politécnico Nacional.

\subsection{Resumen de actividades}
A continuación, se muestra una tabla que resume las principales actividades culturales y deportivas que ofrece la UPIICSA.

\begin{table}[H]
	\centering
	\caption{Actividades culturales y deportivas en la UPIICSA}
	\label{tab:actividades_upiicsa}
	\begin{tabularx}{\textwidth}{|X|X|X|}
		\hline
		\textbf{Tipo de actividad} & \textbf{Ejemplos} & \textbf{Objetivo principal} \\
		\hline
		Culturales & Danza folclórica, teatro, coro, rondalla, fotografía, pintura, guitarra & Desarrollar la creatividad y fomentar la identidad cultural politécnica. \\
		\hline
		Deportivas & Fútbol, básquetbol, voleibol, atletismo, Tae Kwon Do, natación, ajedrez & Promover la salud física, la disciplina y el trabajo en equipo. \\
		\hline
		Recreativas & Acondicionamiento físico, torneos internos, actividades al aire libre & Fomentar la convivencia, la integración y el bienestar estudiantil. \\
		\hline
	\end{tabularx}
\end{table}

\section{Servicios estudiantiles}
La UPIICSA ofrece diversos servicios estudiantiles que tienen como finalidad apoyar la formación integral de los alumnos, facilitando su desarrollo académico, personal y profesional.  
Estos servicios proporcionan recursos, orientación y espacios diseñados para mejorar la experiencia universitaria y promover el bienestar dentro de la comunidad politécnica.

\subsection{Objetivo}
Brindar a los estudiantes los recursos y apoyos necesarios para asegurar su permanencia, rendimiento y éxito académico, así como su bienestar físico y emocional durante su estancia en la institución.

\subsection{Principales servicios}

\subsubsection{Biblioteca}
La biblioteca de la UPIICSA ofrece un extenso acervo bibliográfico en formato físico y digital, especializado en ingeniería, administración, informática, transporte y ciencias sociales.  
Cuenta con áreas de estudio individual y grupal, préstamo de material y acceso a bases de datos académicas del Instituto Politécnico Nacional.

\subsubsection{Becas}
La institución participa en diversos programas de apoyo económico y reconocimiento académico, como las becas institucionales del IPN, las Becas Benito Juárez y los programas de movilidad nacional e internacional.  
Estas becas buscan incentivar la excelencia académica y apoyar a los estudiantes con recursos limitados.

\subsubsection{Tutorías académicas}
El programa de tutorías asigna un profesor tutor a cada grupo o estudiante para ofrecer acompañamiento académico y orientación durante su trayectoria escolar.  
El objetivo es detectar y atender dificultades académicas, personales o vocacionales, promoviendo la continuidad de los estudios.

\subsubsection{Laboratorios y centros de cómputo}
UPIICSA cuenta con laboratorios equipados para las áreas de ingeniería, informática, transporte y administración.  
Los centros de cómputo ofrecen acceso a software especializado, conexión a internet y recursos tecnológicos para la realización de prácticas, proyectos y trabajos académicos.

\subsubsection{Servicio médico}
El servicio médico institucional brinda atención básica, primeros auxilios y orientación preventiva a los estudiantes.  
Además, organiza campañas de salud y bienestar físico en coordinación con el Departamento de Servicios Estudiantiles del IPN.

\subsubsection{Orientación psicológica y vocacional}
La unidad también cuenta con atención psicológica individual y grupal, enfocada en el bienestar emocional del estudiante.  
El servicio de orientación vocacional ayuda a los alumnos a definir sus metas académicas y profesionales de acuerdo con sus intereses y habilidades.

\subsection{Resumen de servicios estudiantiles}
En la siguiente tabla se resumen los principales servicios que ofrece la UPIICSA a su comunidad estudiantil.

\begin{table}[H]
	\centering
	\caption{Servicios estudiantiles ofrecidos por la UPIICSA}
	\label{tab:servicios_upiicsa}
	\begin{tabularx}{\textwidth}{|X|X|X|}
		\hline
		\textbf{Servicio} & \textbf{Descripción} & \textbf{Objetivo principal} \\
		\hline
		Biblioteca & Acervo bibliográfico físico y digital, préstamo de libros, bases de datos académicas. & Facilitar el acceso a la información y al conocimiento especializado. \\
		\hline
		Becas & Programas de apoyo económico y movilidad académica. & Fomentar la permanencia y la excelencia académica. \\
		\hline
		Tutorías académicas & Acompañamiento y asesoría personalizada por profesores tutores. & Apoyar el desarrollo académico y personal del estudiante. \\
		\hline
		Laboratorios y centros de cómputo & Espacios con equipamiento especializado y software técnico. & Fortalecer la formación práctica y tecnológica. \\
		\hline
		Servicio médico & Atención médica básica y programas preventivos de salud. & Promover el bienestar físico y la salud integral. \\
		\hline
		Orientación psicológica y vocacional & Atención emocional y apoyo en la toma de decisiones académicas. & Mejorar la estabilidad emocional y vocacional del estudiante. \\
		\hline
	\end{tabularx}
\end{table}

\section{Rutas de evacuación en UPIICSA}

La Unidad Profesional Interdisciplinaria de Ingeniería y Ciencias Sociales y Administrativas (UPIICSA) cuenta con rutas de evacuación claramente señalizadas en cada uno de sus edificios, con el propósito de garantizar la seguridad de la comunidad estudiantil, docente y administrativa ante cualquier emergencia.

Las rutas de evacuación están diseñadas para guiar a las personas hacia zonas seguras y puntos de reunión previamente establecidos, minimizando el riesgo durante sismos, incendios u otras contingencias. 
Cada pasillo, aula y laboratorio dispone de señalizaciones fotoluminiscentes visibles aun en condiciones de poca luz.

\subsection{Zonas seguras y puntos de reunión}

Cada edificio de UPIICSA cuenta con áreas designadas como zonas seguras, a las cuales se debe dirigir la comunidad durante una evacuación:

\begin{itemize}
	\item \textbf{Edificio de Gobierno:} Las rutas conducen hacia el patio central y la explanada principal frente al edificio.
	\item \textbf{Laboratorios Pesados:} Los usuarios deben dirigirse hacia el área despejada contigua al edificio de Básicas.
	\item \textbf{Laboratorios Ligeros:} Las rutas guían hacia el estacionamiento posterior, que funge como zona de concentración.
	\item \textbf{Edificio de Básicas:} Las salidas principales llevan hacia el área verde lateral y hacia la explanada central.
	\item \textbf{Edificio de Ingeniería:} Los pasillos principales conducen hacia la plaza central frente a los edificios de Gobierno y Culturales.
	\item \textbf{Edificio de Culturales:} La evacuación se realiza hacia el patio central y las zonas abiertas adyacentes a los talleres deportivos.
\end{itemize}

\subsection{Recomendaciones generales de evacuación}

Durante una situación de emergencia, el personal y los estudiantes deben seguir las siguientes indicaciones:

\begin{enumerate}
	\item Mantener la calma y seguir las instrucciones del personal de brigadas de Protección Civil.
	\item No correr, empujar ni gritar durante la evacuación.
	\item No regresar por objetos personales una vez iniciada la salida.
	\item Dirigirse a las zonas seguras más cercanas siguiendo las señales indicativas.
	\item Esperar en el punto de reunión hasta recibir instrucciones de las autoridades correspondientes.
\end{enumerate}

El cumplimiento de estas medidas permite una evacuación ordenada y segura, contribuyendo a la protección de toda la comunidad politécnica.

