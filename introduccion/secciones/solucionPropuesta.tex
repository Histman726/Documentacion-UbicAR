\section{Solución propuesta}

UbicAR es una aplicación diseñada para facilitar la localización y orientación dentro de las instalaciones universitarias mediante el uso de realidad aumentada (AR) y sistemas de navegación interior. Su objetivo principal es ayudar a los estudiantes a encontrar sus salones, academias, cubículos de profesores y demás espacios académicos de manera rápida, visual e intuitiva.

El sistema permitirá al usuario ingresar o vincular su horario de clases, y a partir de ello, mostrarle la ubicación exacta del aula correspondiente dentro del edificio. Además, integrará información adicional como los horarios de atención de los profesores, los espacios de las academias, y las rutas de acceso más convenientes, todo presentado en un entorno aumentado y accesible desde dispositivos móviles.

\subsection{Diferencias respecto al trabajo previo}

A diferencia de las soluciones comerciales y académicas revisadas, UbicAR no se limita únicamente a ofrecer navegación interior o mapas interactivos, sino que integra funciones académicas y administrativas propias del entorno universitario, lo que amplía su utilidad práctica.

\begin{enumerate}
	
	\item Integración con la vida académica
	
	\begin{itemize}
		
		\item Mientras las soluciones previas (como ARway o Navigine) se centran en la orientación espacial, UbicAR enlaza directamente la información de horarios, aulas y docentes, permitiendo al estudiante ubicar sus clases y conocer los datos de contacto y atención de cada profesor.
		
	\end{itemize}
	
	\item Enfoque institucional y personalizado
	
	\begin{itemize}
		
		\item Las aplicaciones comerciales son genéricas y aplicables a diversos entornos (aeropuertos, museos, centros comerciales). En cambio, UbicAR está diseñada específicamente para el contexto universitario, adaptándose a la estructura, edificios y organización interna de una institución educativa.
		
	\end{itemize}
	
	\item Accesibilidad y simplicidad
	
	\begin{itemize}
		
		\item A diferencia de algunos proyectos de investigación que requieren sensores o configuraciones complejas, UbicAR busca aprovechar herramientas accesibles (como cámaras móviles y tecnología ARCore/ARKit), reduciendo costos y facilitando su adopción.
		
	\end{itemize}
	
	\item Visualización contextual mediante AR
	
	\begin{itemize}
		
		\item En lugar de limitarse a un mapa 2D, UbicAR proyectará indicadores virtuales sobre el entorno real, guiando al usuario de forma inmersiva hacia su destino dentro del edificio.
		
	\end{itemize}
	
	\item Extensión de información institucional
	
	\begin{itemize}
		
		\item Además de la navegación, la aplicación incluirá datos sobre ubicación de academias, cubículos de profesores y servicios universitarios, lo cual no es común en las soluciones revisadas.
		
	\end{itemize}
	
\end{enumerate}

\subsection{Conclusión}

En resumen, UbicAR se diferencia por combinar navegación interior con información académica personalizada, utilizando la realidad aumentada como medio de orientación intuitivo y accesible. Su enfoque está en mejorar la experiencia de los estudiantes dentro del campus, optimizando tiempo, ubicaciones y comunicación con el personal docente.