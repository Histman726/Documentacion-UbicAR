\chapter{Introducción}
En el presente capítulo se abordará la problemática de encontrar las academias de los edificios, así como los salones correspondientes en la Unidad Profesional Interdisciplinaria de Ingeniería, Ciencias Sociales y Administrativas.

	\section{Contexto de trabajo}

Encontrar un salón dentro de una universidad puede resultar tedioso, confuso y complicado, incluso cuando se cuenta con un listado de aulas y horarios. En muchas ocasiones, la dificultad radica en no saber con precisión dónde se ubican los salones, especialmente si la institución está conformada por varios edificios.

UbicAR surge como una solución práctica a este problema, ofreciendo una forma sencilla y accesible de localizar el aula correspondiente a cada clase. De esta manera, los estudiantes podrán orientarse fácilmente dentro del campus una vez que dispongan de su horario.

Además, UbicAR proporcionará información sobre la ubicación de las academias en cada edificio, los cubículos de los profesores, sus horarios de atención y los salones donde imparten clase, facilitando así la experiencia de navegación dentro de la universidad.
	\section{Problemática}
En la actualidad, los estudiantes enfrentan diversas dificultades para localizar sus salones dentro de las instalaciones universitarias, especialmente cuando los campus están conformados por varios edificios o áreas con una distribución compleja. Aunque se disponga de listados de horarios y aulas, la falta de una guía visual o de orientación precisa provoca confusión, pérdida de tiempo y retrasos en las actividades académicas.

Asimismo, la ubicación de academias, cubículos de profesores y sus horarios de atención suele ser información dispersa o poco accesible, lo que complica la comunicación y el aprovechamiento de los recursos institucionales.

Por lo tanto, se identifica la necesidad de contar con una herramienta que permita a los estudiantes orientarse fácilmente dentro del campus universitario y acceder de manera rápida y confiable a la información sobre aulas, docentes y espacios académicos.
	\section{Trabajo previo}

Describir el trabajo actual que se ha realizado en el contexto de trabajo del proyecto: Productos comerciales, Tesis, artículos
	\section{Solución propuesta}

UbicAR es una aplicación diseñada para facilitar la localización y orientación dentro de las instalaciones universitarias mediante el uso de realidad aumentada (AR) y sistemas de navegación interior. Su objetivo principal es ayudar a los estudiantes a encontrar sus salones, academias, cubículos de profesores y demás espacios académicos de manera rápida, visual e intuitiva.

El sistema permitirá al usuario ingresar o vincular su horario de clases, y a partir de ello, mostrarle la ubicación exacta del aula correspondiente dentro del edificio. Además, integrará información adicional como los horarios de atención de los profesores, los espacios de las academias, y las rutas de acceso más convenientes, todo presentado en un entorno aumentado y accesible desde dispositivos móviles.

\subsection{Diferencias respecto al trabajo previo}

A diferencia de las soluciones comerciales y académicas revisadas, UbicAR no se limita únicamente a ofrecer navegación interior o mapas interactivos, sino que integra funciones académicas y administrativas propias del entorno universitario, lo que amplía su utilidad práctica.

\begin{enumerate}
	
	\item Integración con la vida académica
	
	\begin{itemize}
		
		\item Mientras las soluciones previas (como ARway o Navigine) se centran en la orientación espacial, UbicAR enlaza directamente la información de horarios, aulas y docentes, permitiendo al estudiante ubicar sus clases y conocer los datos de contacto y atención de cada profesor.
		
	\end{itemize}
	
	\item Enfoque institucional y personalizado
	
	\begin{itemize}
		
		\item Las aplicaciones comerciales son genéricas y aplicables a diversos entornos (aeropuertos, museos, centros comerciales). En cambio, UbicAR está diseñada específicamente para el contexto universitario, adaptándose a la estructura, edificios y organización interna de una institución educativa.
		
	\end{itemize}
	
	\item Accesibilidad y simplicidad
	
	\begin{itemize}
		
		\item A diferencia de algunos proyectos de investigación que requieren sensores o configuraciones complejas, UbicAR busca aprovechar herramientas accesibles (como cámaras móviles y tecnología ARCore/ARKit), reduciendo costos y facilitando su adopción.
		
	\end{itemize}
	
	\item Visualización contextual mediante AR
	
	\begin{itemize}
		
		\item En lugar de limitarse a un mapa 2D, UbicAR proyectará indicadores virtuales sobre el entorno real, guiando al usuario de forma inmersiva hacia su destino dentro del edificio.
		
	\end{itemize}
	
	\item Extensión de información institucional
	
	\begin{itemize}
		
		\item Además de la navegación, la aplicación incluirá datos sobre ubicación de academias, cubículos de profesores y servicios universitarios, lo cual no es común en las soluciones revisadas.
		
	\end{itemize}
	
\end{enumerate}

\subsection{Conclusión}

En resumen, UbicAR se diferencia por combinar navegación interior con información académica personalizada, utilizando la realidad aumentada como medio de orientación intuitivo y accesible. Su enfoque está en mejorar la experiencia de los estudiantes dentro del campus, optimizando tiempo, ubicaciones y comunicación con el personal docente.


	\section{Introducción}

El proyecto \textbf{UbicAR} tiene como objetivo orientar a los usuarios dentro de la UPIICSA, localizando salones y academias de manera rápida mediante una interfaz intuitiva y, potencialmente, el uso de elementos de Realidad Aumentada (AR). En esta sección se presenta el contexto general y la motivación del proyecto, junto con recursos visuales que facilitan su comprensión.

\begin{figure}[H]
    \centering
    \includegraphics[width=0.60\textwidth]{imagenes/1.jpg}
    \caption{Concepto visual de UbicAR: interfaz móvil con elementos de ubicación.}
    \label{fig:concepto-app}
\end{figure}


\begin{figure}[H]
    \centering
    \includegraphics[width=0.30\textwidth]{imagenes/logo-ubicar.png}
    \caption{Identidad visual del proyecto UbicAR.}
    \label{fig:logo-ubicar}
\end{figure}

La motivación surge de la necesidad frecuente de encontrar aulas y academias específicas dentro del campus, un proceso que puede ser tedioso para nuevos estudiantes o visitantes. El sistema propuesto busca reducir tiempos de búsqueda y brindar una experiencia más fluida.

\begin{figure}[H]
    \centering
    \includegraphics[width=0.9\textwidth]{imagenes/mapa-upiicsa.jpg}
    \caption{Mapa general del campus UPIICSA, punto de aplicación del sistema UbicAR.}
    \label{fig:mapa-upiicsa}
\end{figure}

La Figura~\ref{fig:mapa-upiicsa} proporciona el contexto espacial del proyecto (áreas y edificios objetivo), mientras que la Figura~\ref{fig:concepto-app} ilustra el concepto visual de la aplicación y su principal caso de uso: consultar y recibir indicaciones hacia un salón específico.

